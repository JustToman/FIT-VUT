\documentclass[11pt,twocolumn]{article}
\usepackage[a4paper,left=1.5cm,top=2.5cm,text={18cm,25cm}]{geometry}
\usepackage[utf8]{inputenc}
\usepackage[IL2]{fontenc}
\usepackage[czech]{babel}
\usepackage{times}
\usepackage{amsmath}
\usepackage{amsthm}
\usepackage[unicode]{hyperref}
\usepackage{amsfonts}

\newtheorem{definice}{Definice}
\theoremstyle{plain}
\newtheorem{veta}{Věta}

\begin{document}
    \begin{titlepage}
        \begin{center}
            \Huge
                \textsc{Fakulta informačních technologií
                Vysoké~učení technické v~Brně}\\
            \vspace{\stretch{0.3}}
            \LARGE
                Typografie a~publikování -- 2. projekt\\
                Sazba dokumentů a~matematických výrazů \\
            \vspace{\stretch{0.4}}
        \end{center}
        {\Large 2021 \hfill Roman Popelka (xpopel24)}
    \end{titlepage}

\section*{Úvod}
\label{page1}
V~této úloze si vyzkoušíme sazbu titulní strany, mat\-e\-matic\-kých\ vzorců, prostředí a~dalších textových struktur obvyklých pro technicky zaměřené texty (například rovnice \eqref{eq1} nebo Definice \ref{def1} na straně \pageref{page1}). Rovněž si vyzkoušíme používání odkazů \texttt{\textbackslash ref} a~\texttt{\textbackslash pageref}.

Na titulní straně je využito sázení nadpisu podle optického středu s~využitím zlatého řezu. Tento postup byl probírán na přednášce. Dále je použito odřádkování se zadanou relativní velikostí 0.4\,em a 0.3\,em.

V případě, že budete potřebovat vyjádřit matematickou konstrukci nebo symbol a~nebude se Vám dařit jej nalézt v~samotném \LaTeX u, doporučuji prostudovat možnosti balíku maker \AmS-\LaTeX.

\section{Matematický text}
Nejprve se podíváme na sázení matematických symbolů a~výrazů v~plynulém textu včetně sazby definic a~vět s~využitím balíku \texttt{amsthm}. Rovněž použijeme poznámku pod čarou s~použitím příkazu \texttt{\textbackslash footnote}. Někdy je vhodné použít konstrukci \texttt{\textbackslash mbox\string{\string}}, která říká, že text nemá být zalomen.
\begin{definice}\label{def1}
    \emph{Rozšířený zásobníkový automat (RZA)} je definován jako sedmice tvaru $A = (Q,\Sigma,\Gamma,\delta,q_0,Z_0,F)$, kde:
    \begin{itemize}
        \item $Q$ je konečná množina \emph{vnitřních (řídicích) stavů,}
        \item $\Sigma$ je konečná \emph{vstupní abeceda,}
        \item $\Gamma$ je konečná \emph{zásobníková abeceda,}
        \item $\delta$ je \emph{přechodová funkce} $Q \times(\Sigma \cup\{\epsilon\}) \times \Gamma^{*} \rightarrow 2^{Q \times \Gamma^{*}}$,
        \item $q_0 \in Q$ je \emph{počáteční stav}, $Z_0 \in \Gamma$ je \emph{startovací symbol zásobníku} a $F \subseteq Q$ je množina \emph{koncových stavů.}
    \end{itemize}
    
    \emph{Nechť $P = (Q,\Sigma,\Gamma,\delta,q_0,Z_0,F)$ je rozšířený zásobníkový automat.} Konfigurací \emph{nazveme trojici $(q, w,\alpha)\,\in Q \times \Sigma^{*} \times \Gamma^{*}$, kde $q$ je aktuální stav vnitřního řízení, $w$ je dosud nezpracovaná část vstupního řetězce a $\alpha = Z_{i_1}Z_{i_2}\dots Z_{i_k}$ je obsah zásobníku\footnote{$Z_{i_1}$ je vrchol zásobníku}.}
\end{definice}
\subsection{Podsekce obsahující větu a odkaz}
\begin{definice}\label{def2}
    \emph{Řetězec $w$ nad abecedou $\Sigma$ je přijat RZA} $A$~jestliže $(q_0,w,Z_0)\overset{*}{\underset{A}\vdash}(q_F,\epsilon,\gamma)$ pro nějaké $\gamma \in \Gamma^{*}
    $ a $q_F \in F$. Množinu $L(A) = \{w\mid w \text{ je přijat RZA } A\} \subseteq \Sigma^{*}$ nazýváme \emph{jazyk přijímaný RZA} $A$. 
\end{definice}
Nyní si vyzkoušíme sazbu vět a~důkazů opět s~použitím balíku \texttt{amsthm}.
\begin{veta}
    Třída jazyků, které jsou přijímány ZA, odpovídá \emph{bezkontextovým jazykům.}
\end{veta}
\begin{proof}
    V~důkaze vyjdeme z~Definice \ref{def1} a \ref{def2}.
\end{proof}
\section{Rovnice a odkazy}
Složitější matematické formulace sázíme mimo plynulý text. Lze umístit několik výrazů na jeden řádek, ale pak je třeba tyto vhodně oddělit, například příkazem \texttt{\textbackslash quad}.
\\
$$
\sqrt[i]{x_{i}^{3}} \quad \text{kde } x_i \text{ je } i \text{-té sudé číslo splňující}\quad x_{i}^{x_{i}^{i^{2}}+2} \leq y_{i}^{x_{i}^{4}}
$$

V~rovnici \eqref{eq1} jsou využity tři typy závorek s~různou e\-xplicitně definovanou velikostí.
\begin{align}
           \quad x\quad&=\quad\bigg[\Big\{\big[a+b\big] * c\Big\}^{d} \oplus 2\bigg]^{3 / 2}\label{eq1}\\
           \quad y\quad&=\quad\lim _{x \rightarrow \infty} \frac{\frac{1}{\log _{10} x}}{\sin ^{2} x+\cos ^{2} x} \nonumber
\end{align}

V~této větě vidíme, jak vypadá implicitní vysázení limity $\lim _{n \rightarrow \infty} f(n)$ v normálním odstavci textu. Podobně je to i~s~dalšími symboly jako $\prod_{i=1}^{n} 2^{i}$ či ~$\bigcap_{A \in \mathcal{B}}A$. V~případě vzorců $\lim\limits_{n\rightarrow \infty}f(n)$ a~$\prod\limits_{i=1}^n 2^{i}$ jsme si vynutili méně úspornou sazbu příkazem \texttt{\textbackslash limits}.

\begin{equation}
    \quad\int_{b}^{a}g(x)\,dx\quad=\quad -\int\limits_{a}^{b}f(x)\,dx
\end{equation}

\section{Matice}
Pro sázení matic se velmi často používá prostředí \texttt{array} a~závorky (\texttt{\textbackslash left}, \texttt{\textbackslash right}).
$$
\left(
\begin{array}{ccc}
    a-b & \widehat{\xi+\omega} & \pi \\
    \vec{\mathbf{a}} & \overleftrightarrow{AC} & \hat{\beta}\\
\end{array}
\right)
= 1 \Longleftrightarrow \mathcal{Q} = \mathbb{R}
$$
$$
\mathbf{A} = 
\begin{array}{||cccc||}
a_{11} & a_{12} & \dots & a_{1n}\\
a_{21} & a_{22} & \dots & a_{2n}\\
\vdots & \vdots & \ddots & \vdots\\
 a_{m1} & a_{m2} & \dots & a_{mn}
\end{array}
= 
\begin{array}{|cc|}
t & u \\
v & w
\end{array}
= tw - uv
$$
Prostředí \texttt{array} lze úspěšně využít i~jinde.
$$
\binom{n}{k} = 
\left\{
\begin{array}{cl}
     0 & \text{pro } k < 0 \text{ nebo } k > n\\
     \frac{n!}{k!(n-k)!} & \text{pro } 0 \leq k \leq n.
\end{array}
\right.
$$
\end{document}